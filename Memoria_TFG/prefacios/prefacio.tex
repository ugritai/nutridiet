\chapter*{}
%\thispagestyle{empty}
%\cleardoublepage

%\thispagestyle{empty}

%\input{portada/portada_2}



\cleardoublepage
\thispagestyle{empty}

\begin{center}
{\large\bfseries \myTitle}\\
\end{center}
\begin{center}
\myName\\
\end{center}

%\vspace{0.7cm}
\noindent{\textbf{Palabras clave}:  nutrición, planificación dietética, React, FastAPI, MongoDB}\\

\vspace{0.7cm}
\noindent{\textbf{Resumen}}\\

A pesar de la existencia de una amplia variedad de sistemas de información centrados en el ámbito de la nutrición, la mayoría de ellos corresponde a software privativo o no ha sido concebido para su uso directo por parte de profesionales y expertos en nutrición clínica. Esta limitación dificulta su adopción en contextos educativos y sanitarios, donde la accesibilidad, la adaptabilidad y la transparencia tecnológica son factores clave. El presente proyecto busca cubrir esta carencia mediante el diseño y desarrollo de un sistema de información orientado a la creación de dietas personalizadas para uso experto.

Para alcanzar este objetivo general, se han desarrollado una serie de tareas clave. En primer lugar, se llevó a cabo un estudio comparativo de software nutricional libre, centrado en sus funcionalidades y limitaciones. Este análisis permitió detectar carencias comunes y orientar el diseño del sistema hacia necesidades reales del entorno sanitario y educativo.

A continuación, se implementó la conexión con una base de datos nutricional validada, sobre la que se construyó una arquitectura que permite crear dietas personalizadas, aplicar filtros nutricionales inteligentes y realizar un seguimiento individualizado del paciente.

Finalmente, con resultado final el sistema fue preparado para su validación en la Facultad de Farmacia de UGR, como parte de un trabajo colaborativo orientado a su uso en el entorno académico. 
\cleardoublepage


\thispagestyle{empty}


\begin{center}
{\large\bfseries Healthy Nutrition Information System for Professional Use}\\
\end{center}
\begin{center}
Linqi Zhu\\
\end{center}

%\vspace{0.7cm}
\noindent{\textbf{Keywords}: nutrition, diet planning, React, FastAPI, MongoDB}\\

\vspace{0.7cm}
\noindent{\textbf{Abstract}}\\

Despite the wide variety of information systems related to nutrition, most current solutions are either paid (proprietary) or not designed for direct use by nutrition professionals. This makes it difficult to use them in education and healthcare, where easy access, flexibility, and clear technology are very important. This project aims to solve that problem by creating an information system designed to help experts plan personalized diets.

To achieve this general objective, a set of key tasks was carried out. First, a comparative study of free and open-source nutritional software was conducted, analyzing their features and limitations. This evaluation helped identify common shortcomings and guided the system's design toward real needs in clinical and academic settings.

Next, a connection was established with a validated nutritional database, on which an architecture was built to enable personalized diet creation, intelligent nutritional filtering, and individual patient monitoring.

Finally, the system was prepared for validation at the Faculty of Pharmacy (University of Granada), as part of a collaborative initiative aimed at its implementation in academic environments.

\chapter*{}
\thispagestyle{empty}

\noindent\rule[-1ex]{\textwidth}{2pt}\\[4.5ex]

Yo, \textbf{Linqi Zhu}, alumno de la titulación TITULACIÓN de la \textbf{Escuela Técnica Superior
de Ingenierías Informática y de Telecomunicación de la Universidad de Granada}, con DNI X6300759R, autorizo la
ubicación de la siguiente copia de mi Trabajo Fin de Grado en la biblioteca del centro para que pueda ser
consultada por las personas que lo deseen.

\vspace{6cm}

\noindent Fdo: Linqi Zhu

\vspace{2cm}

\begin{flushright}
Granada a 16 de Junio de 2025.
\end{flushright}


\chapter*{}
\thispagestyle{empty}

\noindent\rule[-1ex]{\textwidth}{2pt}\\[4.5ex]

D. \textbf{María José Martín Bautista}, Profesora del Departamento de Ciencias de la Computación e Inteligencia Artificial de la Universidad de Granada.

\vspace{0.5cm}

D. \textbf{Andrea Morales Garzón}, Profesora del Departamento de Ciencias de la Computación e Inteligencia Artificial de la Universidad de Granada.


\vspace{0.5cm}

\textbf{Informan:}

\vspace{0.5cm}

Que el presente trabajo, titulado \textit{\textbf{Sistema de información de nutrición saludable para uso profesional}},
ha sido realizado bajo su supervisión por \textbf{Linqi Zhu}, y autorizamos la defensa de dicho trabajo ante el tribunal
que corresponda.

\vspace{0.5cm}

Y para que conste, expiden y firman el presente informe en Granada a 16 de Junio de 2025 .

\vspace{1cm}

\textbf{Los directores:}

\vspace{5cm}

\noindent \textbf{María José Martín Bautista \ \ \ \ \ Andrea Morales Garzón}

\chapter*{Agradecimientos}
\thispagestyle{empty}

       \vspace{1cm}


A lo largo del desarrollo de este proyecto he contado con el apoyo, la orientación y el acompañamiento de muchas personas a las que me gustaría expresar mi más sincero agradecimiento.

En primer lugar, quiero agradecer especialmente a mis tutoras, María José y Andrea, por sus dedicaciones, paciencia y valiosas sugerencias durante todo el proceso. Desde el inicio del proyecto hemos mantenido reuniones semanales que han sido fundamentales para guiar el trabajo, resolver dudas y tomar decisiones clave en cada etapa. Su acompañamiento cercano ha sido decisivo para poder completar este proyecto con éxito en tan poco tiempo. 

También quiero agradecer a mi familia, por su apoyo incondicional y ánimo constante a lo largo de este proceso. Especialmente a mi hermana Linli, quien me ayudó con la revisión de la memoria, siguiendo todas las versiones y los cambios realizados. Igualmente, agradezco a mi amiga RouYu que me ha acompañado desde el principio y diseñó con cariño el logo de este sistema, aportando su creatividad y tiempo para dar forma visual al proyecto. 

Y por último, a todas las personas que, de forma directa o indirecta, han contribuido al desarrollo de este proyecto con sus comentarios y sugerencias.



